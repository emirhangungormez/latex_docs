\documentclass[a4paper,12pt]{report}

% Paketler
\usepackage[utf8]{inputenc}
\usepackage[T1]{fontenc}
\usepackage[english]{babel}
\usepackage{geometry}
\usepackage{setspace}
\usepackage{graphicx}
\usepackage{titlesec}
\usepackage{fancyhdr}
\usepackage{lipsum}

% Sayfa düzeni
\geometry{a4paper, top=2.5cm, bottom=2.5cm, left=2.5cm, right=2.5cm}

% Başlık biçimi
\titleformat{\chapter}[display]{\normalfont\bfseries\centering}{\chaptertitlename\ \thechapter}{10pt}{\large}

% Sayfa numaraları
\fancypagestyle{plain}{
  \fancyhf{}
  \fancyfoot[C]{\thepage}
  \renewcommand{\headrulewidth}{0pt}
}

% Başlık, yazar ve tarih
\title{Proje Raporu Başlığı}
\author{Adınız Soyadınız}
\date{\today}

\begin{document}

\maketitle

% İçindekiler
\tableofcontents

% Bölüm 1
\chapter{Giriş}
\section{Proje Tanımı}
Bu bölümde projenizin genel tanımını yapabilirsiniz.

\section{Amaç ve Hedefler}
Projenizin amaçlarını ve hedeflerini belirtin.

% Bölüm 2
\chapter{Proje Planı}
\section{Zaman Çizelgesi}
Projenizin zaman çizelgesini burada sunun.

\section{Kaynak Gereksinimleri}
Proje için gerekli kaynakları ve bütçeyi açıklayın.

% Bölüm 3
\chapter{Proje İlerlemesi}
\section{Mevcut Durum}
Projenin şu anki durumu hakkında bilgi verin.

\section{Başarılar ve Sorunlar}
Projenin başarılarını ve karşılaşılan sorunları tartışın.

% Bölüm 4
\chapter{Sonuçlar ve Öneriler}
\section{Sonuçlar}
Projenin sonuçlarına dair bilgi sunun.

\section{Öneriler}
Proje ile ilgili önerilerinizi paylaşın.

% Ekler
\appendix
\chapter{Ekler}
Bu bölümde ek materyalleri ekleyebilirsiniz.

% Kaynaklar
\begin{thebibliography}{9}
\bibitem{example-ref} Örnek bir referans
Bu, bir örnek referanstır.
\end{thebibliography}

\end{document}
