%\documentclass[mathtime]{qjmam}
\documentclass{qjmam}
\usepackage{amssymb}

% If you do not have mathtime fonts please remove and use standard computer modern.

\startpage{1}
\yr{2023}
\vol{1}
\issue{1}

\newtheorem{definition}{Definition}[section]
\newtheorem{lemma}[definition]{Lemma}
\newtheorem{Theorem}[definition]{Theorem}
\newtheorem{proposition}[definition]{Proposition}
\newtheorem{corollary}[definition]{Corollary}
\newtheorem{remark}[definition]{Remark}
\newtheorem{example}[definition]{Example}
\newtheorem{algorithm}[definition]{Algorithm}

%\newtheorem{lemma}{Lemma}
%\newtheorem{example}{Example}
%\newtheorem{proposition}{Proposition}

\DeclareMathAlphabet\mathbit
    \encodingdefault\rmdefault\bfdefault\itdefault
\DeclareOldFontCommand{\bi}{\normalfont\bfseries\itshape}{\mathbit}


\let\bdot\cdot
\def\bdot{\mbox{\bi${\cdot}$}}


\newcommand{\be}{\begin{equation}}
\newcommand{\ee}{\end{equation}}
\def \jt{\!\!\!\!\!&}
\def\fakebold#1{\relax\ifvmode\leavevmode\fi%
\ifmmode%
\setbox0=\hbox{$#1$}%
\else%
\setbox0=\hbox{#1}%
\fi%
\kern-.02em\copy0 \kern-\wd0%
\kern .04em\copy0 \kern-\wd0%
\kern-.0125em\raise.02em\box0%
}%
\newcommand{\bmA}{\mbox{\fakebold{$A$}}}
\newcommand{\bmB}{\mbox{\fakebold{$B$}}}
\newcommand{\bmD}{\mbox{\fakebold{$D$}}}
\newcommand{\bmH}{\mbox{\fakebold{$H$}}}
\newcommand{\bme}{\mbox{\fakebold{$e$}}}
\newcommand{\bmk}{\mbox{\fakebold{$k$}}}
\newcommand{\bmx}{\mbox{\fakebold{$x$}}}
\renewcommand{\.}{\raisebox{.9mm}{.}}
\renewcommand{\geq}{\geqslant}
\renewcommand{\leq}{\leqslant}
%\rule[1mm]{2em}{.15mm}

\begin{document}

\title[{surface~tension~effects~in~a~wedge}] {SENATE BILDIRISI}


\author[bıldırı] {Emirhan G.\and Baturalp G.}

\extraauthor{(Genesis-Being-Existence)}

\received{\textbf{9 Eylul 2023}}


\maketitle

\eqnobysec

\begin{abstract} 

\end{abstract}


\section{Görev Tanımı}
Hafta içerisinde gerçekleştirilecek toplantı nazarında hazırlanılmasını istediğimiz dokümanlar var. İlgili kişilerden; \textit{Proje1}, \textit{Proje2}... için ayrı dokümanlar hazırlanmalı. Doküman aşağıdaki başlıkları içermelidir. \\

\textbf{Proje Başlığı ve Tanımı:} \\
• Proje hedefleri ve amaçları \\
• Projenin neden gerektiği \\
• Projenin kapsamı ve sınırları \\

\textbf{Proje Ekibi:}\\
• İletişim bilgileri\\
• Proje ekibinin organizasyon içindeki konumu (Ekol-içi aktif faaliyet yürüttükleri diğer alanlar)\\

\textbf{Proje Planı:}\\
• Proje başlangıç ve bitiş tarihleri (En önemlisi, net bir şekilde istiyoruz)\\
• Ana milestone'lar ve bu milestone'lara ulaşma planı (Diğer önemli bir konu)\\
• Çalışma paketleri ve görevler\\
• Kaynak gereksinimleri (Yatırımcılardan talep edeceklerimiz)\\
• Bütçe tahmini (Yatırımcılardan talep edeceklerimiz)\\
• Riskler ve risk yönetimi planı (Finansal Projeksiyonu iyi dizayn etmelisiniz)\\

\textbf{Proje İlerleme Durumu:}\\
• Mevcut proje durumu (Halihazır durum ile ilgili ilerleme kaydı ve rapor)\\
• Başarılan kilometre taşları (Projelerde neredeyiz?)\\
• Gecikmeler veya sorunlar (Ekibin eleştirilerine yer veriniz)\\
• Değişiklikler veya güncellemeler (Güncelleme tarihlerine mutlaka yer veriniz)\\

\textbf{İletişim Planı:}\\
• Proje ile ilgili iletişim kuralları ve protokoller (Tüm işi hangi platformda hangi ilkeler doğrultusunda yürütüyorsunuz?)\\
• Toplantılar ve toplantıların düzenlendiği zamanlar (Ne sıklıkla toplantı oluyor? Ne kadar sürüyor? Toplantıların konusu ne oluyor, yalnızca bilgilendirme toplantıları mı yapılıyor yoksa çalışma toplantıları da düzenleniyor mu?)\\

\textbf{Proje Başarı Kriterleri ve Ölçülebilirlik:}\\
• Projenin başarısını değerlendirmek için kullanılacak ölçütler (Proje sizin için ne zaman başarılı olacak? Prototip çıktığında mı, yatırım aldığında mı?)\\
Not: Yazacaklarınız başarı kriterlerinizin nasıl ölçüleceğini belirleyecektir.\\

\textbf{Kaynaklar ve Referanslar:}\\
• Proje ile ilgili kullanılan kaynaklar (Şuan da devops kültürü hakim mi? Ne tür teknolojiler kullanılıyor?)\\

\textbf{Uygulama Tasarımı:}\\
• Geliştirilen tasarım\\
• Planlanan tasarım\\
• Rakip proje tasarımları hakkında bilgi\\

\textbf{Onay ve Imza:}\\
• İlgili kişiler ile gizlilik anlaşması kabulü (Önemli)\\

\textit{\textbf{Not}: Doküman tarihini 11.09.23 olarak belirleyiniz.}

\end{document}
